% IMPORTANT: PLEASE USE XeLaTeX FOR TYPESETTING
\documentclass[dark]{sintefbeamer}
\usepackage{xeCJK}
\usepackage{graphicx}
\usepackage{color}
\usepackage{tabularx}
\usepackage{booktabs}


% \usepackage{natbib}
% meta-data
\title{TEXT-TO-3D}
\subtitle{AI-Generated Content}
\author{Enyun Xuan}
\date{2024 年 2 月 27 日}
\titlebackground{images/background}

% document body
\setbeamertemplate{bibliography item}{}
% \bibliographystyle{apalike} % 这个样式支持作者-年份格式
\setbeamertemplate{footline}[frame number]
\begin{document}

\maketitle

\begin{frame}{近期工作}
  \framesubtitle{2.19-2.26}

  近期学习了\cite{chenSurvey3DGaussian2024}这篇综述中,对于3DGS在AI-Generated Content(AIGC)领域中的Text-to-3D。

  \begin{itemize}
    \item 扩散模型(Diffusion Model)
    \item 生成模型(Generated Model)
  \end{itemize}

\end{frame}

\section{AI-Generated Content (AIGC)}

\begin{frame}[fragile]{Diffusion Model}{\thesection \, \secname}

  根据热力学的扩散原理,\cite{sohl-dicksteinDeepUnsupervisedLearning2015}、\cite{hoDenoisingDiffusionProbabilistic2020}提出了新的生成模型——扩散模型(Diffusion Model)。

  \begin{figure}
    \includegraphics[width=0.8\textwidth]{../storage/89FWG783/image.png}
      \end{figure}
\end{frame}

\begin{frame}[fragile]{Diffusion Model}{\thesection \, \secname}

  \begin{figure}
    \includegraphics[width=\textwidth]{../storage/8YGCTKD7/image.png}
      \end{figure}
\end{frame}

\begin{frame}[fragile]{Stable Diffusion}{\thesection \, \secname}

  Stable Diffusion\cite{rombachHighResolutionImageSynthesis2022}模型是AIGC领域的重要成果,可以根据TEXT来生成内容。

  \begin{figure}
    \includegraphics[width=0.8\textwidth]{../storage/AMR4HR56/image.png}
  \end{figure}

\end{frame}

\begin{frame}{Stable Diffusion}

  在训练过程中加入了\textcolor{red}{Condition}。

  \begin{figure}
    \includegraphics[width=0.8\textwidth]{../storage/3GPT8PCS/image.png}
  \end{figure}

\end{frame}

\section{TEXT-TO-3D}

\begin{frame}[fragile]{DreamFusion}

DreamFusion\cite{pooleDreamFusionTextto3DUsing2022}仅使用2D图片生成3D模型。

    \begin{figure}
      \includegraphics[width=0.9\textwidth]{../storage/G7NK7YVD/image.png}
    \end{figure}

\end{frame}

\begin{frame}{Score Distillation Sampling}

  在\textcolor{red}{Parameter space}进行取样,而不是Pixel space。

  \begin{figure}
    \includegraphics[width=\textwidth]{../storage/8D2ZNU77/image.png}
  \end{figure}

  与NeRF相结合的桥梁。

\end{frame}

\begin{frame}[fragile]{DreamFusion}

  DreamFusion的贡献:
  
  \begin{itemize}
    \item 仅使用2D数据集;
    \item 结合Mip-NeRF;
    \item 分数蒸馏取样——SCORE DISTILLATION SAMPLING(SDS)
  \end{itemize}
  
\end{frame}

\begin{frame}[fragile]{DreamFusion}

  出现的问题:
  \begin{columns}
    \begin{column}{0.5\textwidth}
      \begin{itemize}
        \item 训练时间长;
        \item Janus Problem
      \end{itemize}
    \end{column}
  
    \begin{column}{0.5\textwidth}
      \begin{figure}
        \includegraphics[width=\textwidth]{../storage/PPFED733/image.png}
      \end{figure}
    \end{column}
  \end{columns}
  
\end{frame}

\begin{frame}[fragile]{GSGEN}

  GSGEN\cite{chenTextto3DUsingGaussian2023}用3DGS替换NeRF,达到了新的SOTA。

  \begin{figure}
    \includegraphics[width=\textwidth]{../storage/5HQH6LIT/image.png}
  \end{figure}

\end{frame}

\begin{frame}[fragile]{GSGEN}

  克服了隐式表达中几何形状坍塌的问题。

  \begin{figure}
    \includegraphics[width=\textwidth]{../storage/DK67IDUV/image.png}
  \end{figure}

\end{frame}

\begin{frame}[fragile]{Text-to-PointCloud}

  Point-E\cite{nicholPointESystemGenerating2022}从prompt生成点云,作为3D prior。

  \begin{figure}
    \includegraphics[width=\textwidth]{../storage/XGI4B68J/image.png}
  \end{figure}

\end{frame}

\begin{frame}[fragile]{Compactness-based Densification}
  \begin{columns}
    \begin{column}{0.5\textwidth}
      保证几何结构。
      \begin{itemize}
        \item 删减高丝球
        \item 插入高丝球
      \end{itemize}
  
    \end{column}
    \begin{column}{0.5\textwidth}
      \begin{figure}
        \includegraphics[width=\textwidth]{../storage/RA95MT87/image.png}
      \end{figure}
    \end{column}
  \end{columns}
\end{frame}

\begin{frame}[fragile]{Plan}
  
  后续计划:

  \begin{itemize}
    \item 阅读并理解以上内容的数学推导;
    \item 运行开源代码;
    \item 挖掘改进空间。
  \end{itemize}
 
\end{frame}

\backmatter

\begin{frame}[t, allowframebreaks]{References}{\,}
\framesubtitle{\quad}
\tiny
  \bibliographystyle{apalike}
  \bibliography{references}
\end{frame}


\end{document}
